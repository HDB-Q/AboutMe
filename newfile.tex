% Options for packages loaded elsewhere
\PassOptionsToPackage{unicode}{hyperref}
\PassOptionsToPackage{hyphens}{url}
\PassOptionsToPackage{dvipsnames,svgnames,x11names}{xcolor}
%
\documentclass[
  letterpaper,
  DIV=11,
  numbers=noendperiod,
  oneside,
  11pt]{scrartcl}

\usepackage{amsmath,amssymb}
\usepackage{iftex}
\ifPDFTeX
  \usepackage[T1]{fontenc}
  \usepackage[utf8]{inputenc}
  \usepackage{textcomp} % provide euro and other symbols
\else % if luatex or xetex
  \usepackage{unicode-math}
  \defaultfontfeatures{Scale=MatchLowercase}
  \defaultfontfeatures[\rmfamily]{Ligatures=TeX,Scale=1}
\fi
\usepackage[]{libertinus}
\ifPDFTeX\else  
    % xetex/luatex font selection
\fi
% Use upquote if available, for straight quotes in verbatim environments
\IfFileExists{upquote.sty}{\usepackage{upquote}}{}
\IfFileExists{microtype.sty}{% use microtype if available
  \usepackage[]{microtype}
  \UseMicrotypeSet[protrusion]{basicmath} % disable protrusion for tt fonts
}{}
\makeatletter
\@ifundefined{KOMAClassName}{% if non-KOMA class
  \IfFileExists{parskip.sty}{%
    \usepackage{parskip}
  }{% else
    \setlength{\parindent}{0pt}
    \setlength{\parskip}{6pt plus 2pt minus 1pt}}
}{% if KOMA class
  \KOMAoptions{parskip=half}}
\makeatother
\usepackage{xcolor}
\usepackage[margin = 1in]{geometry}
\setlength{\emergencystretch}{3em} % prevent overfull lines
\setcounter{secnumdepth}{5}
% Make \paragraph and \subparagraph free-standing
\makeatletter
\ifx\paragraph\undefined\else
  \let\oldparagraph\paragraph
  \renewcommand{\paragraph}{
    \@ifstar
      \xxxParagraphStar
      \xxxParagraphNoStar
  }
  \newcommand{\xxxParagraphStar}[1]{\oldparagraph*{#1}\mbox{}}
  \newcommand{\xxxParagraphNoStar}[1]{\oldparagraph{#1}\mbox{}}
\fi
\ifx\subparagraph\undefined\else
  \let\oldsubparagraph\subparagraph
  \renewcommand{\subparagraph}{
    \@ifstar
      \xxxSubParagraphStar
      \xxxSubParagraphNoStar
  }
  \newcommand{\xxxSubParagraphStar}[1]{\oldsubparagraph*{#1}\mbox{}}
  \newcommand{\xxxSubParagraphNoStar}[1]{\oldsubparagraph{#1}\mbox{}}
\fi
\makeatother


\providecommand{\tightlist}{%
  \setlength{\itemsep}{0pt}\setlength{\parskip}{0pt}}\usepackage{longtable,booktabs,array}
\usepackage{calc} % for calculating minipage widths
% Correct order of tables after \paragraph or \subparagraph
\usepackage{etoolbox}
\makeatletter
\patchcmd\longtable{\par}{\if@noskipsec\mbox{}\fi\par}{}{}
\makeatother
% Allow footnotes in longtable head/foot
\IfFileExists{footnotehyper.sty}{\usepackage{footnotehyper}}{\usepackage{footnote}}
\makesavenoteenv{longtable}
\usepackage{graphicx}
\makeatletter
\def\maxwidth{\ifdim\Gin@nat@width>\linewidth\linewidth\else\Gin@nat@width\fi}
\def\maxheight{\ifdim\Gin@nat@height>\textheight\textheight\else\Gin@nat@height\fi}
\makeatother
% Scale images if necessary, so that they will not overflow the page
% margins by default, and it is still possible to overwrite the defaults
% using explicit options in \includegraphics[width, height, ...]{}
\setkeys{Gin}{width=\maxwidth,height=\maxheight,keepaspectratio}
% Set default figure placement to htbp
\makeatletter
\def\fps@figure{htbp}
\makeatother

\KOMAoption{captions}{tableheading}
\makeatletter
\@ifpackageloaded{caption}{}{\usepackage{caption}}
\AtBeginDocument{%
\ifdefined\contentsname
  \renewcommand*\contentsname{Table of contents}
\else
  \newcommand\contentsname{Table of contents}
\fi
\ifdefined\listfigurename
  \renewcommand*\listfigurename{List of Figures}
\else
  \newcommand\listfigurename{List of Figures}
\fi
\ifdefined\listtablename
  \renewcommand*\listtablename{List of Tables}
\else
  \newcommand\listtablename{List of Tables}
\fi
\ifdefined\figurename
  \renewcommand*\figurename{Figure}
\else
  \newcommand\figurename{Figure}
\fi
\ifdefined\tablename
  \renewcommand*\tablename{Table}
\else
  \newcommand\tablename{Table}
\fi
}
\@ifpackageloaded{float}{}{\usepackage{float}}
\floatstyle{ruled}
\@ifundefined{c@chapter}{\newfloat{codelisting}{h}{lop}}{\newfloat{codelisting}{h}{lop}[chapter]}
\floatname{codelisting}{Listing}
\newcommand*\listoflistings{\listof{codelisting}{List of Listings}}
\captionsetup{labelsep=colon}
\makeatother
\makeatletter
\makeatother
\makeatletter
\@ifpackageloaded{caption}{}{\usepackage{caption}}
\@ifpackageloaded{subcaption}{}{\usepackage{subcaption}}
\makeatother

\ifLuaTeX
\usepackage[bidi=basic]{babel}
\else
\usepackage[bidi=default]{babel}
\fi
\babelprovide[main,import]{english}
% get rid of language-specific shorthands (see #6817):
\let\LanguageShortHands\languageshorthands
\def\languageshorthands#1{}
\ifLuaTeX
  \usepackage{selnolig}  % disable illegal ligatures
\fi
\usepackage[]{biblatex}
\usepackage{bookmark}

\IfFileExists{xurl.sty}{\usepackage{xurl}}{} % add URL line breaks if available
\urlstyle{same} % disable monospaced font for URLs
\hypersetup{
  pdftitle={newfiletitle},
  pdfauthor={Erik Erhardt},
  pdflang={en},
  colorlinks=true,
  linkcolor={blue},
  filecolor={Maroon},
  citecolor={Blue},
  urlcolor={Blue},
  pdfcreator={LaTeX via pandoc}}


\title{newfiletitle}
\usepackage{etoolbox}
\makeatletter
\providecommand{\subtitle}[1]{% add subtitle to \maketitle
  \apptocmd{\@title}{\par {\large #1 \par}}{}{}
}
\makeatother
\subtitle{Subtitle}
\author{Erik Erhardt}
\date{08/22/2024 ~10:34:24 PM -0400}

\begin{document}
\maketitle
\begin{abstract}
Document abstract may be somewhat long.
\end{abstract}

\renewcommand*\contentsname{Contents}
{
\hypersetup{linkcolor=}
\setcounter{tocdepth}{4}
\tableofcontents
}
\listoffigures
\listoftables

\newpage{}

\section{Executive summary}\label{sec-Executive_summary}

\newpage{}

\section{Introduction}\label{sec-Introduction}

\subsection{Background/rationale}\label{backgroundrationale}

\begin{itemize}
\tightlist
\item
  Explain the scientific background and rationale for the investigation
  being reported
\end{itemize}

\subsection{Objectives}\label{objectives}

\begin{itemize}
\tightlist
\item
  State specific objectives, including any prespecified hypotheses
\end{itemize}

\subsection{Literature Review:}\label{literature-review}

\begin{itemize}
\tightlist
\item
  Review of relevant prior research and theories.
\item
  Identification of gaps in existing knowledge.
\end{itemize}

\newpage{}

\section{Methods}\label{sec-Methods}

\subsection{Study design}\label{study-design}

\begin{itemize}
\tightlist
\item
  Present key elements of study design early in the paper
\end{itemize}

\subsection{Setting}\label{setting}

\begin{itemize}
\tightlist
\item
  Describe the setting, locations, and relevant dates, including periods
  of recruitment, exposure, follow-up, and data collection
\end{itemize}

\subsection{Participants}\label{participants}

\begin{itemize}
\tightlist
\item
  Cohort study --- Give the eligibility criteria, and the sources and
  methods of selection of participants. Describe methods of follow-up
\item
  Case-control study --- Give the eligibility criteria, and the sources
  and methods of case ascertainment and control selection. Give the
  rationale for the choice of cases and controls
\item
  Cross-sectional study --- Give the eligibility criteria, and the
  sources and methods of selection of participants
\item
  Cohort study --- For matched studies, give matching criteria and
  number of exposed and unexposed
\item
  Case-control study --- For matched studies, give matching criteria and
  the number of controls per case
\end{itemize}

\subsection{Variables}\label{variables}

\begin{itemize}
\tightlist
\item
  Clearly define all outcomes, exposures, predictors, potential
  confounders, and effect modifiers. Give diagnostic criteria, if
  applicable
\end{itemize}

\subsection{Data sources/measurement}\label{data-sourcesmeasurement}

\begin{itemize}
\tightlist
\item
  Give information separately for cases and controls in case-control
  studies and, if applicable, for exposed and unexposed groups in cohort
  and cross-sectional studies.
\item
  For each variable of interest, give sources of data and details of
  methods of assessment (measurement). Describe comparability of
  assessment methods if there is more than one group
\end{itemize}

\subsection{Bias}\label{bias}

\begin{itemize}
\tightlist
\item
  Describe any efforts to address potential sources of bias
\end{itemize}

\subsection{Study size}\label{study-size}

\begin{itemize}
\tightlist
\item
  Explain how the study size was arrived at
\end{itemize}

\subsection{Quantitative variables}\label{quantitative-variables}

\begin{itemize}
\tightlist
\item
  Explain how quantitative variables were handled in the analyses. If
  applicable, describe which groupings were chosen and why
\end{itemize}

\subsection{Statistical methods}\label{statistical-methods}

\begin{itemize}
\tightlist
\item
  Describe all statistical methods, including those used to control for
  confounding
\item
  Describe any methods used to examine subgroups and interactions
\item
  Explain how missing data were addressed
\item
  Cohort study --- If applicable, explain how loss to follow-up was
  addressed
\item
  Case-control study --- If applicable, explain how matching of cases
  and controls was addressed
\item
  Cross-sectional study --- If applicable, describe analytical methods
  taking account of sampling strategy
\item
  Describe any sensitivity analyses
\end{itemize}

\newpage{}


\printbibliography[title=Results]



\end{document}
